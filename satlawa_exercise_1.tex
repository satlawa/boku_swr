% Options for packages loaded elsewhere
\PassOptionsToPackage{unicode}{hyperref}
\PassOptionsToPackage{hyphens}{url}
%
\documentclass[
]{article}
\usepackage{lmodern}
\usepackage{amssymb,amsmath}
\usepackage{ifxetex,ifluatex}
\ifnum 0\ifxetex 1\fi\ifluatex 1\fi=0 % if pdftex
  \usepackage[T1]{fontenc}
  \usepackage[utf8]{inputenc}
  \usepackage{textcomp} % provide euro and other symbols
\else % if luatex or xetex
  \usepackage{unicode-math}
  \defaultfontfeatures{Scale=MatchLowercase}
  \defaultfontfeatures[\rmfamily]{Ligatures=TeX,Scale=1}
\fi
% Use upquote if available, for straight quotes in verbatim environments
\IfFileExists{upquote.sty}{\usepackage{upquote}}{}
\IfFileExists{microtype.sty}{% use microtype if available
  \usepackage[]{microtype}
  \UseMicrotypeSet[protrusion]{basicmath} % disable protrusion for tt fonts
}{}
\makeatletter
\@ifundefined{KOMAClassName}{% if non-KOMA class
  \IfFileExists{parskip.sty}{%
    \usepackage{parskip}
  }{% else
    \setlength{\parindent}{0pt}
    \setlength{\parskip}{6pt plus 2pt minus 1pt}}
}{% if KOMA class
  \KOMAoptions{parskip=half}}
\makeatother
\usepackage{xcolor}
\IfFileExists{xurl.sty}{\usepackage{xurl}}{} % add URL line breaks if available
\IfFileExists{bookmark.sty}{\usepackage{bookmark}}{\usepackage{hyperref}}
\hypersetup{
  pdftitle={Statistics with R - Exercise 1},
  pdfauthor={Philipp Satlawa - h0640348},
  hidelinks,
  pdfcreator={LaTeX via pandoc}}
\urlstyle{same} % disable monospaced font for URLs
\usepackage[margin=1in]{geometry}
\usepackage{color}
\usepackage{fancyvrb}
\newcommand{\VerbBar}{|}
\newcommand{\VERB}{\Verb[commandchars=\\\{\}]}
\DefineVerbatimEnvironment{Highlighting}{Verbatim}{commandchars=\\\{\}}
% Add ',fontsize=\small' for more characters per line
\usepackage{framed}
\definecolor{shadecolor}{RGB}{248,248,248}
\newenvironment{Shaded}{\begin{snugshade}}{\end{snugshade}}
\newcommand{\AlertTok}[1]{\textcolor[rgb]{0.94,0.16,0.16}{#1}}
\newcommand{\AnnotationTok}[1]{\textcolor[rgb]{0.56,0.35,0.01}{\textbf{\textit{#1}}}}
\newcommand{\AttributeTok}[1]{\textcolor[rgb]{0.77,0.63,0.00}{#1}}
\newcommand{\BaseNTok}[1]{\textcolor[rgb]{0.00,0.00,0.81}{#1}}
\newcommand{\BuiltInTok}[1]{#1}
\newcommand{\CharTok}[1]{\textcolor[rgb]{0.31,0.60,0.02}{#1}}
\newcommand{\CommentTok}[1]{\textcolor[rgb]{0.56,0.35,0.01}{\textit{#1}}}
\newcommand{\CommentVarTok}[1]{\textcolor[rgb]{0.56,0.35,0.01}{\textbf{\textit{#1}}}}
\newcommand{\ConstantTok}[1]{\textcolor[rgb]{0.00,0.00,0.00}{#1}}
\newcommand{\ControlFlowTok}[1]{\textcolor[rgb]{0.13,0.29,0.53}{\textbf{#1}}}
\newcommand{\DataTypeTok}[1]{\textcolor[rgb]{0.13,0.29,0.53}{#1}}
\newcommand{\DecValTok}[1]{\textcolor[rgb]{0.00,0.00,0.81}{#1}}
\newcommand{\DocumentationTok}[1]{\textcolor[rgb]{0.56,0.35,0.01}{\textbf{\textit{#1}}}}
\newcommand{\ErrorTok}[1]{\textcolor[rgb]{0.64,0.00,0.00}{\textbf{#1}}}
\newcommand{\ExtensionTok}[1]{#1}
\newcommand{\FloatTok}[1]{\textcolor[rgb]{0.00,0.00,0.81}{#1}}
\newcommand{\FunctionTok}[1]{\textcolor[rgb]{0.00,0.00,0.00}{#1}}
\newcommand{\ImportTok}[1]{#1}
\newcommand{\InformationTok}[1]{\textcolor[rgb]{0.56,0.35,0.01}{\textbf{\textit{#1}}}}
\newcommand{\KeywordTok}[1]{\textcolor[rgb]{0.13,0.29,0.53}{\textbf{#1}}}
\newcommand{\NormalTok}[1]{#1}
\newcommand{\OperatorTok}[1]{\textcolor[rgb]{0.81,0.36,0.00}{\textbf{#1}}}
\newcommand{\OtherTok}[1]{\textcolor[rgb]{0.56,0.35,0.01}{#1}}
\newcommand{\PreprocessorTok}[1]{\textcolor[rgb]{0.56,0.35,0.01}{\textit{#1}}}
\newcommand{\RegionMarkerTok}[1]{#1}
\newcommand{\SpecialCharTok}[1]{\textcolor[rgb]{0.00,0.00,0.00}{#1}}
\newcommand{\SpecialStringTok}[1]{\textcolor[rgb]{0.31,0.60,0.02}{#1}}
\newcommand{\StringTok}[1]{\textcolor[rgb]{0.31,0.60,0.02}{#1}}
\newcommand{\VariableTok}[1]{\textcolor[rgb]{0.00,0.00,0.00}{#1}}
\newcommand{\VerbatimStringTok}[1]{\textcolor[rgb]{0.31,0.60,0.02}{#1}}
\newcommand{\WarningTok}[1]{\textcolor[rgb]{0.56,0.35,0.01}{\textbf{\textit{#1}}}}
\usepackage{graphicx,grffile}
\makeatletter
\def\maxwidth{\ifdim\Gin@nat@width>\linewidth\linewidth\else\Gin@nat@width\fi}
\def\maxheight{\ifdim\Gin@nat@height>\textheight\textheight\else\Gin@nat@height\fi}
\makeatother
% Scale images if necessary, so that they will not overflow the page
% margins by default, and it is still possible to overwrite the defaults
% using explicit options in \includegraphics[width, height, ...]{}
\setkeys{Gin}{width=\maxwidth,height=\maxheight,keepaspectratio}
% Set default figure placement to htbp
\makeatletter
\def\fps@figure{htbp}
\makeatother
\setlength{\emergencystretch}{3em} % prevent overfull lines
\providecommand{\tightlist}{%
  \setlength{\itemsep}{0pt}\setlength{\parskip}{0pt}}
\setcounter{secnumdepth}{-\maxdimen} % remove section numbering

\title{Statistics with R - Exercise 1}
\author{Philipp Satlawa - h0640348}
\date{06/11/2020}

\begin{document}
\maketitle

This document contains the answered questions of exercise 1 for the
course ``Statistics with R''.

\begin{center}\rule{0.5\linewidth}{0.5pt}\end{center}

\hypertarget{task-1---sequences}{%
\subsection{Task 1 - Sequences}\label{task-1---sequences}}

Creation and comparison of vectors.

\begin{enumerate}
\def\labelenumi{\arabic{enumi}.}
\tightlist
\item
  Create a vector \texttt{x} that contains the sequence of even numbers
  from 0 to 40 (\texttt{x} \(\in\) \rm I\!R)
\end{enumerate}

\begin{Shaded}
\begin{Highlighting}[]
\NormalTok{(x <-}\StringTok{ }\KeywordTok{seq}\NormalTok{(}\DataTypeTok{from =} \DecValTok{0}\NormalTok{, }\DataTypeTok{to =} \DecValTok{40}\NormalTok{, }\DataTypeTok{by =} \DecValTok{2}\NormalTok{))}
\end{Highlighting}
\end{Shaded}

\begin{verbatim}
##  [1]  0  2  4  6  8 10 12 14 16 18 20 22 24 26 28 30 32 34 36 38 40
\end{verbatim}

\begin{enumerate}
\def\labelenumi{\arabic{enumi}.}
\setcounter{enumi}{1}
\tightlist
\item
  Create a vector \texttt{y}, which contains the elements of vector
  \texttt{x} but in \emph{random} order (\texttt{y} \(\in\) \texttt{x})
\end{enumerate}

\begin{Shaded}
\begin{Highlighting}[]
\CommentTok{# set seed to produce reproducible results}
\KeywordTok{set.seed}\NormalTok{(}\DecValTok{10}\NormalTok{)}
\CommentTok{# take random samples from the elements of vector x}
\NormalTok{(y <-}\StringTok{ }\KeywordTok{sample}\NormalTok{(x))}
\end{Highlighting}
\end{Shaded}

\begin{verbatim}
##  [1] 20 16 18 30 22 12 14 10 34  4 28 36 24  2 40  8 32 38  6  0 26
\end{verbatim}

\begin{enumerate}
\def\labelenumi{\arabic{enumi}.}
\setcounter{enumi}{2}
\tightlist
\item
  The values of \texttt{x} and \texttt{y} agree on the following
  positions:
\end{enumerate}

\begin{Shaded}
\begin{Highlighting}[]
\CommentTok{# number of positions the two vectors agree on}
\KeywordTok{length}\NormalTok{(}\KeywordTok{which}\NormalTok{(x }\OperatorTok{==}\StringTok{ }\NormalTok{y))}
\end{Highlighting}
\end{Shaded}

\begin{verbatim}
## [1] 2
\end{verbatim}

\begin{Shaded}
\begin{Highlighting}[]
\CommentTok{# indices (positions) on which the values of the two vectors agree on}
\KeywordTok{which}\NormalTok{(x }\OperatorTok{==}\StringTok{ }\NormalTok{y)}
\end{Highlighting}
\end{Shaded}

\begin{verbatim}
## [1] 13 17
\end{verbatim}

\begin{center}\rule{0.5\linewidth}{0.5pt}\end{center}

\hypertarget{task-2---sequences}{%
\subsection{Task 2 - Sequences}\label{task-2---sequences}}

Verification of three approximation formulas for \(\pi\). Since the
calculation of \(\pi\) by the following formulas includes infinite sums
or products, we use a large \(n = 10000\).

Creating vector \texttt{i} as a sequence from 1 to 100000 (\texttt{x}
\(\in\) \rm I\!R).

\begin{Shaded}
\begin{Highlighting}[]
\CommentTok{# create vector i}
\NormalTok{i <-}\StringTok{ }\DecValTok{1}\OperatorTok{:}\DecValTok{10000}
\end{Highlighting}
\end{Shaded}

\begin{enumerate}
\def\labelenumi{\arabic{enumi}.}
\tightlist
\item
  John Wallis (1616-1703):
\end{enumerate}

\begin{Shaded}
\begin{Highlighting}[]
\CommentTok{# calculation of pi with the formula of John Wallis}
\KeywordTok{prod}\NormalTok{((}\DecValTok{2}\OperatorTok{*}\NormalTok{i}\OperatorTok{/}\NormalTok{(}\DecValTok{2}\OperatorTok{*}\NormalTok{i}\DecValTok{-1}\NormalTok{)) }\OperatorTok{*}\StringTok{ }\NormalTok{(}\DecValTok{2}\OperatorTok{*}\NormalTok{i}\OperatorTok{/}\NormalTok{(}\DecValTok{2}\OperatorTok{*}\NormalTok{i}\OperatorTok{+}\DecValTok{1}\NormalTok{)))}\OperatorTok{*}\DecValTok{2}
\end{Highlighting}
\end{Shaded}

\begin{verbatim}
## [1] 3.141514
\end{verbatim}

\begin{enumerate}
\def\labelenumi{\arabic{enumi}.}
\setcounter{enumi}{1}
\tightlist
\item
  Gottfried Leibnitz (1646-1716):
\end{enumerate}

\begin{Shaded}
\begin{Highlighting}[]
\CommentTok{# calculation of pi with the formula of Gottfried Leibnitz}
\KeywordTok{sum}\NormalTok{(((}\OperatorTok{-}\DecValTok{1}\NormalTok{)}\OperatorTok{**}\NormalTok{(i}\OperatorTok{+}\DecValTok{1}\NormalTok{))}\OperatorTok{/}\NormalTok{(}\DecValTok{2}\OperatorTok{*}\NormalTok{i}\DecValTok{-1}\NormalTok{))}\OperatorTok{*}\DecValTok{4}
\end{Highlighting}
\end{Shaded}

\begin{verbatim}
## [1] 3.141493
\end{verbatim}

\begin{enumerate}
\def\labelenumi{\arabic{enumi}.}
\setcounter{enumi}{2}
\tightlist
\item
  Leonhard Euler (1707-1783):
\end{enumerate}

\begin{Shaded}
\begin{Highlighting}[]
\CommentTok{# calculation of pi with the formula of Leonhard Euler}
\KeywordTok{sqrt}\NormalTok{(}\KeywordTok{sum}\NormalTok{(}\DecValTok{1}\OperatorTok{/}\NormalTok{i}\OperatorTok{**}\DecValTok{2}\NormalTok{)}\OperatorTok{*}\DecValTok{6}\NormalTok{)}
\end{Highlighting}
\end{Shaded}

\begin{verbatim}
## [1] 3.141497
\end{verbatim}

To compute the smallest relative deviation from \(\pi\) given the vector
\texttt{i}, we need \(\pi\)'s current best approximation. Acording to
ikipedia (2020) pi's current best approximation is
3.1415926535897932384626433.

\begin{Shaded}
\begin{Highlighting}[]
\CommentTok{# store pi's current best approximation for comparison}
\NormalTok{pi <-}\StringTok{ }\FloatTok{3.1415926535897932384626433}

\CommentTok{# calculate the difference between John Wallis (1616-1703) and current pi}
\NormalTok{(diff_JW <-}\StringTok{ }\KeywordTok{abs}\NormalTok{(}\KeywordTok{prod}\NormalTok{((}\DecValTok{2}\OperatorTok{*}\NormalTok{i}\OperatorTok{/}\NormalTok{(}\DecValTok{2}\OperatorTok{*}\NormalTok{i}\DecValTok{-1}\NormalTok{)) }\OperatorTok{*}\StringTok{ }\NormalTok{(}\DecValTok{2}\OperatorTok{*}\NormalTok{i}\OperatorTok{/}\NormalTok{(}\DecValTok{2}\OperatorTok{*}\NormalTok{i}\OperatorTok{+}\DecValTok{1}\NormalTok{)))}\OperatorTok{*}\DecValTok{2} \OperatorTok{-}\StringTok{ }\NormalTok{pi))}
\end{Highlighting}
\end{Shaded}

\begin{verbatim}
## [1] 7.853491e-05
\end{verbatim}

\begin{Shaded}
\begin{Highlighting}[]
\CommentTok{# calculate the difference between Gottfried Leibnitz (1646-1716) and current pi}
\NormalTok{(diff_GL <-}\StringTok{ }\KeywordTok{abs}\NormalTok{(}\KeywordTok{sum}\NormalTok{(((}\OperatorTok{-}\DecValTok{1}\NormalTok{)}\OperatorTok{**}\NormalTok{(i}\OperatorTok{+}\DecValTok{1}\NormalTok{))}\OperatorTok{/}\NormalTok{(}\DecValTok{2}\OperatorTok{*}\NormalTok{i}\DecValTok{-1}\NormalTok{))}\OperatorTok{*}\DecValTok{4} \OperatorTok{-}\StringTok{ }\NormalTok{pi))}
\end{Highlighting}
\end{Shaded}

\begin{verbatim}
## [1] 1e-04
\end{verbatim}

\begin{Shaded}
\begin{Highlighting}[]
\CommentTok{# calculate the difference between Leonhard Euler (1707-1783) and current pi}
\NormalTok{(diff_LE <-}\StringTok{ }\KeywordTok{abs}\NormalTok{(}\KeywordTok{sqrt}\NormalTok{(}\KeywordTok{sum}\NormalTok{(}\DecValTok{1}\OperatorTok{/}\NormalTok{i}\OperatorTok{**}\DecValTok{2}\NormalTok{)}\OperatorTok{*}\DecValTok{6}\NormalTok{) }\OperatorTok{-}\StringTok{ }\NormalTok{pi))}
\end{Highlighting}
\end{Shaded}

\begin{verbatim}
## [1] 9.548964e-05
\end{verbatim}

\begin{Shaded}
\begin{Highlighting}[]
\CommentTok{# get the absolute difference of the formula with the smallest absolute difference}
\KeywordTok{min}\NormalTok{(diff_JW, diff_GL, diff_LE)}
\end{Highlighting}
\end{Shaded}

\begin{verbatim}
## [1] 7.853491e-05
\end{verbatim}

Hance, the approximation formula for \(\pi\) with the smallest relative
deviation for \(n = 10000\) is John Wallis (1616-1703) formula.

Wikipedia (2020),
\url{https://en.wikipedia.org/wiki/Approximations_of_\%CF\%80}

\begin{center}\rule{0.5\linewidth}{0.5pt}\end{center}

\hypertarget{task-3---vectors}{%
\subsection{Task 3 - Vectors}\label{task-3---vectors}}

Creation and manipulation of vectors.

\begin{enumerate}
\def\labelenumi{\arabic{enumi}.}
\tightlist
\item
  Create vector \texttt{x} containing the sequence 1 to 100 (\texttt{x}
  \(\in\)
  \rm I\!R) and vector `y` containing a sample of size n = 70 of the sequence 1 to 150 (`y` $\in$ \rm I\!R).
\end{enumerate}

\begin{Shaded}
\begin{Highlighting}[]
\CommentTok{# create vector x containing the sequence 1 to 100 (natural numbers)}
\NormalTok{(x <-}\StringTok{ }\DecValTok{1}\OperatorTok{:}\DecValTok{100}\NormalTok{)}
\end{Highlighting}
\end{Shaded}

\begin{verbatim}
##   [1]   1   2   3   4   5   6   7   8   9  10  11  12  13  14  15  16  17  18
##  [19]  19  20  21  22  23  24  25  26  27  28  29  30  31  32  33  34  35  36
##  [37]  37  38  39  40  41  42  43  44  45  46  47  48  49  50  51  52  53  54
##  [55]  55  56  57  58  59  60  61  62  63  64  65  66  67  68  69  70  71  72
##  [73]  73  74  75  76  77  78  79  80  81  82  83  84  85  86  87  88  89  90
##  [91]  91  92  93  94  95  96  97  98  99 100
\end{verbatim}

\begin{Shaded}
\begin{Highlighting}[]
\CommentTok{# set seed to produce reproducible results}
\KeywordTok{set.seed}\NormalTok{(}\DecValTok{10}\NormalTok{)}
\CommentTok{# create vector y containing a sample of size n = 70 of the sequence 1 to 150 (natural numbers)}
\NormalTok{(y <-}\StringTok{ }\KeywordTok{sample}\NormalTok{(}\DecValTok{1}\OperatorTok{:}\DecValTok{150}\NormalTok{, }\DecValTok{70}\NormalTok{, }\DataTypeTok{replace =} \OtherTok{TRUE}\NormalTok{))}
\end{Highlighting}
\end{Shaded}

\begin{verbatim}
##  [1] 137  74 112  72  88  15 143  74  24  13  95 136 110   7  86  82  29  29 121
## [20]  92  50 109 101 122  33 135  68  93 114  88  51  32  11  79  92  91  42  78
## [39]  13 105 144 117  26  89  48  15 110  24  61 132  14  35  10  74  58 144  15
## [58]  31 138 101 101 109  39 118  89  18 131  42 138  79
\end{verbatim}

\begin{enumerate}
\def\labelenumi{\arabic{enumi}.}
\setcounter{enumi}{1}
\tightlist
\item
  Determine the amount of elements that are contained in \texttt{x} but
  not in \texttt{y}.
\end{enumerate}

\begin{Shaded}
\begin{Highlighting}[]
\CommentTok{# elements of x not contained in y}
\NormalTok{(}\KeywordTok{setdiff}\NormalTok{(x,y))}
\end{Highlighting}
\end{Shaded}

\begin{verbatim}
##  [1]   1   2   3   4   5   6   8   9  12  16  17  19  20  21  22  23  25  27  28
## [20]  30  34  36  37  38  40  41  43  44  45  46  47  49  52  53  54  55  56  57
## [39]  59  60  62  63  64  65  66  67  69  70  71  73  75  76  77  80  81  83  84
## [58]  85  87  90  94  96  97  98  99 100
\end{verbatim}

\begin{Shaded}
\begin{Highlighting}[]
\CommentTok{# number of elements of x not contained in y}
\NormalTok{(}\KeywordTok{length}\NormalTok{(}\KeywordTok{setdiff}\NormalTok{(x,y)))}
\end{Highlighting}
\end{Shaded}

\begin{verbatim}
## [1] 66
\end{verbatim}

\begin{enumerate}
\def\labelenumi{\arabic{enumi}.}
\setcounter{enumi}{2}
\tightlist
\item
  Check for duplicate elements in \texttt{y} and depending if there are
  duplicate elements or not create a different \texttt{z}.
\end{enumerate}

\begin{Shaded}
\begin{Highlighting}[]
\CommentTok{# check if there are duplicate elements in y}
\ControlFlowTok{if}\NormalTok{(}\KeywordTok{length}\NormalTok{(y[}\KeywordTok{duplicated}\NormalTok{(y)]) }\OperatorTok{>}\StringTok{ }\DecValTok{0}\NormalTok{) \{}
  \CommentTok{# create a new vector z containing the duplicate elements of y}
\NormalTok{  z <-}\StringTok{ }\NormalTok{y[}\KeywordTok{duplicated}\NormalTok{(y)]}
\NormalTok{\} }\ControlFlowTok{else}\NormalTok{ \{}
  \CommentTok{# create a new vector z ac a copy of y}
\NormalTok{  z <-}\StringTok{ }\NormalTok{y}
\NormalTok{\}}
\end{Highlighting}
\end{Shaded}

\begin{enumerate}
\def\labelenumi{\arabic{enumi}.}
\setcounter{enumi}{3}
\tightlist
\item
  Determine the number of elements of \texttt{z} that are multiples of
  3.
\end{enumerate}

\begin{Shaded}
\begin{Highlighting}[]
\CommentTok{# calculate the number of elements of z that are multiples of 3}
\KeywordTok{length}\NormalTok{(z[z }\OperatorTok\StringTok{ }\DecValTok{3} \OperatorTok{==}\StringTok{ }\DecValTok{0}\NormalTok{])}
\end{Highlighting}
\end{Shaded}

\begin{verbatim}
## [1] 6
\end{verbatim}

\begin{enumerate}
\def\labelenumi{\arabic{enumi}.}
\setcounter{enumi}{4}
\tightlist
\item
  Revert the \texttt{y} without using the function \texttt{rev()}.
\end{enumerate}

\begin{Shaded}
\begin{Highlighting}[]
\CommentTok{# revert vector y}
\NormalTok{y[}\KeywordTok{length}\NormalTok{(y)}\OperatorTok{:}\DecValTok{1}\NormalTok{]}
\end{Highlighting}
\end{Shaded}

\begin{verbatim}
##  [1]  79 138  42 131  18  89 118  39 109 101 101 138  31  15 144  58  74  10  35
## [20]  14 132  61  24 110  15  48  89  26 117 144 105  13  78  42  91  92  79  11
## [39]  32  51  88 114  93  68 135  33 122 101 109  50  92 121  29  29  82  86   7
## [58] 110 136  95  13  24  74 143  15  88  72 112  74 137
\end{verbatim}

\begin{center}\rule{0.5\linewidth}{0.5pt}\end{center}

\hypertarget{task-4---point-estimation}{%
\subsection{Task 4 - Point Estimation}\label{task-4---point-estimation}}

Assuming a normally distributed population, we create random sample and
estimate \(\mu\) and \(\sigma^{2}\) for this sample.

\begin{enumerate}
\def\labelenumi{\arabic{enumi}.}
\tightlist
\item
  Draw a reproducible sample of size \(n = 30\) from a normal
  distribution with \(\mu=5\) and \(\sigma^{2}=4\).
\end{enumerate}

\begin{Shaded}
\begin{Highlighting}[]
\CommentTok{# set seed to produce reproducible results}
\KeywordTok{set.seed}\NormalTok{(}\DecValTok{10}\NormalTok{)}
\CommentTok{# draw random sample with n = 30, mu = 5, sigma^2 = 4}
\NormalTok{(x <-}\StringTok{ }\KeywordTok{rnorm}\NormalTok{(}\DecValTok{30}\NormalTok{, }\DataTypeTok{mean =} \DecValTok{5}\NormalTok{, }\DataTypeTok{sd =} \KeywordTok{sqrt}\NormalTok{(}\DecValTok{4}\NormalTok{)))}
\end{Highlighting}
\end{Shaded}

\begin{verbatim}
##  [1] 5.0374923 4.6314949 2.2573389 3.8016646 5.5890903 5.7795886 2.5838476
##  [8] 4.2726480 1.7466546 4.4870432 7.2035590 6.5115630 4.5235329 6.9748894
## [15] 6.4827803 5.1786945 3.0901123 4.6096992 6.8510425 5.9659570 3.8073787
## [22] 0.6294263 3.6502681 0.7618776 2.4696040 4.2526769 3.6248891 3.2556823
## [29] 4.7964780 4.4924389
\end{verbatim}

2.Estimate \(\mu\) and \(\sigma^{2}\) on the basis of your sample using
the formulas to estimate the population mean \(\mu\) with the sample
mean \(\overline{x}\) and the population variance \(\sigma^{2}\) with
the empirical variance \(s^2\), without using the functions
\texttt{mean()}, \texttt{var()} and \texttt{sd()}.

\begin{Shaded}
\begin{Highlighting}[]
\CommentTok{# calculation of mean}
\NormalTok{(}\KeywordTok{sum}\NormalTok{(x)}\OperatorTok{/}\KeywordTok{length}\NormalTok{(x))}
\end{Highlighting}
\end{Shaded}

\begin{verbatim}
## [1] 4.310647
\end{verbatim}

\begin{Shaded}
\begin{Highlighting}[]
\CommentTok{# calculation of variance}
\NormalTok{(}\DecValTok{1}\OperatorTok{/}\NormalTok{(}\KeywordTok{length}\NormalTok{(x)}\OperatorTok{-}\DecValTok{1}\NormalTok{) }\OperatorTok{*}\StringTok{ }\KeywordTok{sum}\NormalTok{((x}\OperatorTok{-}\KeywordTok{mean}\NormalTok{(x))}\OperatorTok{**}\DecValTok{2}\NormalTok{))}
\end{Highlighting}
\end{Shaded}

\begin{verbatim}
## [1] 3.00578
\end{verbatim}

\begin{Shaded}
\begin{Highlighting}[]
\CommentTok{# calculation of standard deviation}
\NormalTok{(}\KeywordTok{sqrt}\NormalTok{(}\DecValTok{1}\OperatorTok{/}\NormalTok{(}\KeywordTok{length}\NormalTok{(x)}\OperatorTok{-}\DecValTok{1}\NormalTok{) }\OperatorTok{*}\StringTok{ }\KeywordTok{sum}\NormalTok{((x}\OperatorTok{-}\KeywordTok{mean}\NormalTok{(x))}\OperatorTok{**}\DecValTok{2}\NormalTok{)))}
\end{Highlighting}
\end{Shaded}

\begin{verbatim}
## [1] 1.733718
\end{verbatim}

\begin{enumerate}
\def\labelenumi{\arabic{enumi}.}
\setcounter{enumi}{2}
\tightlist
\item
  Compare your results with the output of the functions \texttt{mean()}
  and \texttt{var()}.
\end{enumerate}

\begin{Shaded}
\begin{Highlighting}[]
\CommentTok{# compute the mean with inbuilt function mean()}
\NormalTok{(}\KeywordTok{mean}\NormalTok{(x))}
\end{Highlighting}
\end{Shaded}

\begin{verbatim}
## [1] 4.310647
\end{verbatim}

\begin{Shaded}
\begin{Highlighting}[]
\CommentTok{# check if the results of the inbuilt function mean() and the formula for calculating the mean are the same }
\ControlFlowTok{if}\NormalTok{(}\KeywordTok{round}\NormalTok{(}\KeywordTok{sum}\NormalTok{(x)}\OperatorTok{/}\KeywordTok{length}\NormalTok{(x),}\DecValTok{5}\NormalTok{) }\OperatorTok{==}\StringTok{ }\KeywordTok{round}\NormalTok{(}\KeywordTok{mean}\NormalTok{(x),}\DecValTok{5}\NormalTok{)) \{}
\NormalTok{  (}\StringTok{"same"}\NormalTok{)}
\NormalTok{\} }\ControlFlowTok{else}\NormalTok{ \{}
\NormalTok{  (}\StringTok{"diffrent"}\NormalTok{)}
\NormalTok{\}  }
\end{Highlighting}
\end{Shaded}

\begin{verbatim}
## [1] "same"
\end{verbatim}

\begin{Shaded}
\begin{Highlighting}[]
\CommentTok{# compute the variance with inbuilt function var()}
\NormalTok{(}\KeywordTok{var}\NormalTok{(x))}
\end{Highlighting}
\end{Shaded}

\begin{verbatim}
## [1] 3.00578
\end{verbatim}

\begin{Shaded}
\begin{Highlighting}[]
\CommentTok{# check if the results of the inbuilt function var() and the formula for calculating the variance are the same }
\ControlFlowTok{if}\NormalTok{(}\KeywordTok{round}\NormalTok{((}\DecValTok{1}\OperatorTok{/}\NormalTok{(}\KeywordTok{length}\NormalTok{(x)}\OperatorTok{-}\DecValTok{1}\NormalTok{) }\OperatorTok{*}\StringTok{ }\KeywordTok{sum}\NormalTok{((x}\OperatorTok{-}\KeywordTok{mean}\NormalTok{(x))}\OperatorTok{**}\DecValTok{2}\NormalTok{)),}\DecValTok{5}\NormalTok{) }\OperatorTok{==}\StringTok{ }\KeywordTok{round}\NormalTok{(}\KeywordTok{var}\NormalTok{(x),}\DecValTok{5}\NormalTok{)) \{}
\NormalTok{  (}\StringTok{"same"}\NormalTok{)}
\NormalTok{\} }\ControlFlowTok{else}\NormalTok{ \{}
\NormalTok{  (}\StringTok{"diffrent"}\NormalTok{)}
\NormalTok{\}}
\end{Highlighting}
\end{Shaded}

\begin{verbatim}
## [1] "same"
\end{verbatim}

\begin{Shaded}
\begin{Highlighting}[]
\CommentTok{# compute the standard variation with inbuilt function sd()}
\NormalTok{(}\KeywordTok{sd}\NormalTok{(x))}
\end{Highlighting}
\end{Shaded}

\begin{verbatim}
## [1] 1.733718
\end{verbatim}

\begin{Shaded}
\begin{Highlighting}[]
\CommentTok{# check if the results of the inbuilt function sd() and the formula for calculating the variance are the same }
\ControlFlowTok{if}\NormalTok{(}\KeywordTok{round}\NormalTok{(}\KeywordTok{sqrt}\NormalTok{(}\DecValTok{1}\OperatorTok{/}\NormalTok{(}\KeywordTok{length}\NormalTok{(x)}\OperatorTok{-}\DecValTok{1}\NormalTok{) }\OperatorTok{*}\StringTok{ }\KeywordTok{sum}\NormalTok{((x}\OperatorTok{-}\KeywordTok{mean}\NormalTok{(x))}\OperatorTok{**}\DecValTok{2}\NormalTok{)),}\DecValTok{5}\NormalTok{) }\OperatorTok{==}\StringTok{ }\KeywordTok{round}\NormalTok{(}\KeywordTok{sd}\NormalTok{(x),}\DecValTok{5}\NormalTok{)) \{}
\NormalTok{  (}\StringTok{"same"}\NormalTok{)}
\NormalTok{\} }\ControlFlowTok{else}\NormalTok{ \{}
\NormalTok{  (}\StringTok{"diffrent"}\NormalTok{)}
\NormalTok{\}}
\end{Highlighting}
\end{Shaded}

\begin{verbatim}
## [1] "same"
\end{verbatim}

\begin{enumerate}
\def\labelenumi{\arabic{enumi}.}
\setcounter{enumi}{3}
\tightlist
\item
  Are your estimates close to the population values? Repeat the steps 1
  and 3 from above with a sample of size n = 3000. What do we learn?
\end{enumerate}

\begin{Shaded}
\begin{Highlighting}[]
\CommentTok{# set seed to produce reproducible results}
\KeywordTok{set.seed}\NormalTok{(}\DecValTok{10}\NormalTok{)}
\CommentTok{# draw random sample with n = 3000, mu = 5, sigma^2 = 4}
\NormalTok{x <-}\StringTok{ }\KeywordTok{rnorm}\NormalTok{(}\DecValTok{3000}\NormalTok{, }\DataTypeTok{mean =} \DecValTok{5}\NormalTok{, }\DataTypeTok{sd =} \KeywordTok{sqrt}\NormalTok{(}\DecValTok{4}\NormalTok{))}

\CommentTok{# calculation of mean}
\NormalTok{(}\KeywordTok{sum}\NormalTok{(x)}\OperatorTok{/}\KeywordTok{length}\NormalTok{(x))}
\end{Highlighting}
\end{Shaded}

\begin{verbatim}
## [1] 5.002634
\end{verbatim}

\begin{Shaded}
\begin{Highlighting}[]
\CommentTok{# calculation of variance}
\NormalTok{(}\DecValTok{1}\OperatorTok{/}\NormalTok{(}\KeywordTok{length}\NormalTok{(x)}\OperatorTok{-}\DecValTok{1}\NormalTok{) }\OperatorTok{*}\StringTok{ }\KeywordTok{sum}\NormalTok{((x}\OperatorTok{-}\KeywordTok{mean}\NormalTok{(x))}\OperatorTok{**}\DecValTok{2}\NormalTok{))}
\end{Highlighting}
\end{Shaded}

\begin{verbatim}
## [1] 4.143197
\end{verbatim}

\begin{Shaded}
\begin{Highlighting}[]
\CommentTok{# calculation of standard deviation}
\NormalTok{(}\KeywordTok{sqrt}\NormalTok{(}\DecValTok{1}\OperatorTok{/}\NormalTok{(}\KeywordTok{length}\NormalTok{(x)}\OperatorTok{-}\DecValTok{1}\NormalTok{) }\OperatorTok{*}\StringTok{ }\KeywordTok{sum}\NormalTok{((x}\OperatorTok{-}\KeywordTok{mean}\NormalTok{(x))}\OperatorTok{**}\DecValTok{2}\NormalTok{)))}
\end{Highlighting}
\end{Shaded}

\begin{verbatim}
## [1] 2.035484
\end{verbatim}

\begin{Shaded}
\begin{Highlighting}[]
\CommentTok{# compute the mean with inbuilt function mean()}
\NormalTok{(}\KeywordTok{mean}\NormalTok{(x))}
\end{Highlighting}
\end{Shaded}

\begin{verbatim}
## [1] 5.002634
\end{verbatim}

\begin{Shaded}
\begin{Highlighting}[]
\CommentTok{# check if the results of the inbuilt function mean() and the formula for calculating the mean are the same }
\ControlFlowTok{if}\NormalTok{(}\KeywordTok{round}\NormalTok{(}\KeywordTok{sum}\NormalTok{(x)}\OperatorTok{/}\KeywordTok{length}\NormalTok{(x),}\DecValTok{5}\NormalTok{) }\OperatorTok{==}\StringTok{ }\KeywordTok{round}\NormalTok{(}\KeywordTok{mean}\NormalTok{(x),}\DecValTok{5}\NormalTok{)) \{}
\NormalTok{  (}\StringTok{"same"}\NormalTok{)}
\NormalTok{\} }\ControlFlowTok{else}\NormalTok{ \{}
\NormalTok{  (}\StringTok{"diffrent"}\NormalTok{)}
\NormalTok{\}  }
\end{Highlighting}
\end{Shaded}

\begin{verbatim}
## [1] "same"
\end{verbatim}

\begin{Shaded}
\begin{Highlighting}[]
\CommentTok{# compute the variance with inbuilt function var()}
\NormalTok{(}\KeywordTok{var}\NormalTok{(x))}
\end{Highlighting}
\end{Shaded}

\begin{verbatim}
## [1] 4.143197
\end{verbatim}

\begin{Shaded}
\begin{Highlighting}[]
\CommentTok{# check if the results of the inbuilt function var() and the formula for calculating the variance are the same }
\ControlFlowTok{if}\NormalTok{(}\KeywordTok{round}\NormalTok{((}\DecValTok{1}\OperatorTok{/}\NormalTok{(}\KeywordTok{length}\NormalTok{(x)}\OperatorTok{-}\DecValTok{1}\NormalTok{) }\OperatorTok{*}\StringTok{ }\KeywordTok{sum}\NormalTok{((x}\OperatorTok{-}\KeywordTok{mean}\NormalTok{(x))}\OperatorTok{**}\DecValTok{2}\NormalTok{)),}\DecValTok{5}\NormalTok{) }\OperatorTok{==}\StringTok{ }\KeywordTok{round}\NormalTok{(}\KeywordTok{var}\NormalTok{(x),}\DecValTok{5}\NormalTok{)) \{}
\NormalTok{  (}\StringTok{"same"}\NormalTok{)}
\NormalTok{\} }\ControlFlowTok{else}\NormalTok{ \{}
\NormalTok{  (}\StringTok{"diffrent"}\NormalTok{)}
\NormalTok{\}}
\end{Highlighting}
\end{Shaded}

\begin{verbatim}
## [1] "same"
\end{verbatim}

\begin{Shaded}
\begin{Highlighting}[]
\CommentTok{# compute the standard variation with inbuilt function sd()}
\NormalTok{(}\KeywordTok{sd}\NormalTok{(x))}
\end{Highlighting}
\end{Shaded}

\begin{verbatim}
## [1] 2.035484
\end{verbatim}

\begin{Shaded}
\begin{Highlighting}[]
\CommentTok{# check if the results of the inbuilt function sd() and the formula for calculating the variance are the same }
\ControlFlowTok{if}\NormalTok{(}\KeywordTok{round}\NormalTok{(}\KeywordTok{sqrt}\NormalTok{(}\DecValTok{1}\OperatorTok{/}\NormalTok{(}\KeywordTok{length}\NormalTok{(x)}\OperatorTok{-}\DecValTok{1}\NormalTok{) }\OperatorTok{*}\StringTok{ }\KeywordTok{sum}\NormalTok{((x}\OperatorTok{-}\KeywordTok{mean}\NormalTok{(x))}\OperatorTok{**}\DecValTok{2}\NormalTok{)),}\DecValTok{5}\NormalTok{) }\OperatorTok{==}\StringTok{ }\KeywordTok{round}\NormalTok{(}\KeywordTok{sd}\NormalTok{(x),}\DecValTok{5}\NormalTok{)) \{}
\NormalTok{  (}\StringTok{"same"}\NormalTok{)}
\NormalTok{\} }\ControlFlowTok{else}\NormalTok{ \{}
\NormalTok{  (}\StringTok{"diffrent"}\NormalTok{)}
\NormalTok{\}}
\end{Highlighting}
\end{Shaded}

\begin{verbatim}
## [1] "same"
\end{verbatim}

The conclusion of the above calculations shows, the higher the sample
size \(n\) the smaller is the deviation between the sample mean
\(\overline{x}\) and the population mean \(\mu\). The same applies for
the variance and standard deviation.

\begin{center}\rule{0.5\linewidth}{0.5pt}\end{center}

\hypertarget{task-5---interval-estimation}{%
\subsection{Task 5 - Interval
Estimation}\label{task-5---interval-estimation}}

Calculate the confidence intervals for the mean and the variance.

\begin{enumerate}
\def\labelenumi{\arabic{enumi}.}
\tightlist
\item
  Draw a reproducible sample of size \(n = 30\) from a normal
  distribution with \(\mu=5\) and \(\sigma^{2}=4\).
\end{enumerate}

\begin{Shaded}
\begin{Highlighting}[]
\CommentTok{# set seed to produce reproducible results}
\KeywordTok{set.seed}\NormalTok{(}\DecValTok{10}\NormalTok{)}
\CommentTok{# draw random sample}
\NormalTok{(x <-}\StringTok{ }\KeywordTok{rnorm}\NormalTok{(}\DecValTok{30}\NormalTok{, }\DataTypeTok{mean =} \DecValTok{5}\NormalTok{, }\DataTypeTok{sd =} \KeywordTok{sqrt}\NormalTok{(}\DecValTok{4}\NormalTok{)))}
\end{Highlighting}
\end{Shaded}

\begin{verbatim}
##  [1] 5.0374923 4.6314949 2.2573389 3.8016646 5.5890903 5.7795886 2.5838476
##  [8] 4.2726480 1.7466546 4.4870432 7.2035590 6.5115630 4.5235329 6.9748894
## [15] 6.4827803 5.1786945 3.0901123 4.6096992 6.8510425 5.9659570 3.8073787
## [22] 0.6294263 3.6502681 0.7618776 2.4696040 4.2526769 3.6248891 3.2556823
## [29] 4.7964780 4.4924389
\end{verbatim}

\begin{enumerate}
\def\labelenumi{\arabic{enumi}.}
\setcounter{enumi}{1}
\tightlist
\item
  Calculate a confidence interval for \(\mu\) and \(\sigma^{2}\) for
  \(\alpha = 0.05\) (hence the confidence level is
  \(1 - \alpha1 = 0.95\)). Inbuilt functions such as mean(), sd() and
  var() are allowed. looked up in sktiptum t(29;0.975) = 2.045
\end{enumerate}

\begin{Shaded}
\begin{Highlighting}[]
\CommentTok{# looked up the t-value in the skript t(29;0.975) = 2.045}
\CommentTok{# get t-value for n-1 = 29; 1-alpha = 0.975 from an inbuilt R function}
\NormalTok{(t <-}\StringTok{ }\KeywordTok{qt}\NormalTok{(}\FloatTok{0.975}\NormalTok{,}\DataTypeTok{df=}\DecValTok{29}\NormalTok{))}
\end{Highlighting}
\end{Shaded}

\begin{verbatim}
## [1] 2.04523
\end{verbatim}

\begin{Shaded}
\begin{Highlighting}[]
\CommentTok{# lower bound mean}
\NormalTok{(low_bound_mean <-}\StringTok{ }\KeywordTok{mean}\NormalTok{(x) }\OperatorTok{-}\StringTok{ }\NormalTok{t }\OperatorTok{*}\StringTok{ }\KeywordTok{sd}\NormalTok{(x)}\OperatorTok{/}\KeywordTok{sqrt}\NormalTok{(}\KeywordTok{length}\NormalTok{(x)))}
\end{Highlighting}
\end{Shaded}

\begin{verbatim}
## [1] 3.663266
\end{verbatim}

\begin{Shaded}
\begin{Highlighting}[]
\CommentTok{# upper bound mean}
\NormalTok{(up_bound_mean <-}\StringTok{ }\KeywordTok{mean}\NormalTok{(x) }\OperatorTok{+}\StringTok{ }\NormalTok{t }\OperatorTok{*}\StringTok{ }\KeywordTok{sd}\NormalTok{(x)}\OperatorTok{/}\KeywordTok{sqrt}\NormalTok{(}\KeywordTok{length}\NormalTok{(x)))}
\end{Highlighting}
\end{Shaded}

\begin{verbatim}
## [1] 4.958028
\end{verbatim}

\begin{Shaded}
\begin{Highlighting}[]
\CommentTok{# get chi-value for n-1 = 29; 1-alpha = 0.975}
\NormalTok{(}\KeywordTok{qchisq}\NormalTok{(}\FloatTok{0.975}\NormalTok{, }\DataTypeTok{df=}\DecValTok{29}\NormalTok{))}
\end{Highlighting}
\end{Shaded}

\begin{verbatim}
## [1] 45.72229
\end{verbatim}

\begin{Shaded}
\begin{Highlighting}[]
\CommentTok{# lower bound standard dev}
\NormalTok{(low_bound_var <-}\StringTok{ }\NormalTok{((}\KeywordTok{length}\NormalTok{(x)}\OperatorTok{-}\DecValTok{1}\NormalTok{) }\OperatorTok{*}\StringTok{ }\KeywordTok{var}\NormalTok{(x)) }\OperatorTok{/}\StringTok{ }\KeywordTok{qchisq}\NormalTok{(}\FloatTok{0.975}\NormalTok{, }\DataTypeTok{df=}\DecValTok{29}\NormalTok{))}
\end{Highlighting}
\end{Shaded}

\begin{verbatim}
## [1] 1.906458
\end{verbatim}

\begin{Shaded}
\begin{Highlighting}[]
\CommentTok{# upper bound standard dev}
\NormalTok{(up_bound_var <-}\StringTok{ }\NormalTok{((}\KeywordTok{length}\NormalTok{(x)}\OperatorTok{-}\DecValTok{1}\NormalTok{) }\OperatorTok{*}\StringTok{ }\KeywordTok{var}\NormalTok{(x)) }\OperatorTok{/}\StringTok{ }\KeywordTok{qchisq}\NormalTok{(}\FloatTok{0.025}\NormalTok{, }\DataTypeTok{df=}\DecValTok{29}\NormalTok{))}
\end{Highlighting}
\end{Shaded}

\begin{verbatim}
## [1] 5.431995
\end{verbatim}

\begin{enumerate}
\def\labelenumi{\arabic{enumi}.}
\setcounter{enumi}{2}
\tightlist
\item
  Determine if true parameters lie in the confidence interval.
\end{enumerate}

\begin{Shaded}
\begin{Highlighting}[]
\CommentTok{# check if the mean lies within the confidence interval of the mean}
\ControlFlowTok{if}\NormalTok{((}\KeywordTok{mean}\NormalTok{(x) }\OperatorTok{>}\StringTok{ }\NormalTok{low_bound_mean) }\OperatorTok{&}\StringTok{ }\NormalTok{(}\KeywordTok{mean}\NormalTok{(x) }\OperatorTok{<}\StringTok{ }\NormalTok{up_bound_mean))\{}
\NormalTok{  (}\StringTok{"the mean lies within the confidence interval"}\NormalTok{)}
\NormalTok{\} }\ControlFlowTok{else}\NormalTok{\{}
\NormalTok{  (}\StringTok{"the mean lies outside the confidence interval"}\NormalTok{)}
\NormalTok{\}}
\end{Highlighting}
\end{Shaded}

\begin{verbatim}
## [1] "the mean lies within the confidence interval"
\end{verbatim}

\begin{Shaded}
\begin{Highlighting}[]
\CommentTok{# check if the variance lies within the confidence interval of the variance}
\ControlFlowTok{if}\NormalTok{((}\KeywordTok{var}\NormalTok{(x) }\OperatorTok{>}\StringTok{ }\NormalTok{low_bound_var) }\OperatorTok{&}\StringTok{ }\NormalTok{(}\KeywordTok{var}\NormalTok{(x) }\OperatorTok{<}\StringTok{ }\NormalTok{up_bound_var))\{}
\NormalTok{  (}\StringTok{"the variation lies within the confidence interval"}\NormalTok{)}
\NormalTok{\} }\ControlFlowTok{else}\NormalTok{\{}
\NormalTok{  (}\StringTok{"the variation lies outside the confidence interval"}\NormalTok{)}
\NormalTok{\}}
\end{Highlighting}
\end{Shaded}

\begin{verbatim}
## [1] "the variation lies within the confidence interval"
\end{verbatim}

Yes, in our case the true parameters lie within our confidence
intervals. This is true for both the mean and the variance. The mean and
the variance are in within the confidence intervals with a
5\%probability of error.

\end{document}
